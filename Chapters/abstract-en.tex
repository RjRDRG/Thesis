%!TEX root = ../template.tex
%%%%%%%%%%%%%%%%%%%%%%%%%%%%%%%%%%%%%%%%%%%%%%%%%%%%%%%%%%%%%%%%%%%%
%% abstrac-en.tex
%% NOVA thesis document file
%%
%% Abstract in English([^%]*)
%%%%%%%%%%%%%%%%%%%%%%%%%%%%%%%%%%%%%%%%%%%%%%%%%%%%%%%%%%%%%%%%%%%%

\typeout{NT FILE abstrac-en.tex}%

In contrast to traditional system designs, microservice architectures provide greater
scalability, availability, and delivery by separating the elements of a large project into independent entities linked through a network (services), however they offer no
mechanisms for safely evolving and discontinuing functionalities of services.

Changing the definition of an element in a regular program can be accomplished quickly with the aid of automated tools.
Distributed systems do not have tools that are comparable to those used in centralized systems.
In the absence of tools that alleviate this problem,
developers are left with the burden of manually tracking down and resolving problems caused by uncontrolled updates.

Whereas traditional approaches ensure that microservices are behaving properly by validating their behavior through empirical tests,
our solution seeks to supplement the conventional approach by providing mechanisms that support the validation of deployment operations and the evolution of microservice interfaces.

We present a microservice management system that verifies the safety of modifications to service interfaces
and allows for interface evolution using runtime-generated proxies that dynamically convert data sent between services to the format expected by static service code,
thereby relieving the developer of the need to manually adapt either new or existing services.

% Palavras-chave do resumo em Inglês
\begin{keywords}
    microservices, software evolution, service compatibility
\end{keywords} 
