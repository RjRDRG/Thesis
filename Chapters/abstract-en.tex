%!TEX root = ../template.tex
%%%%%%%%%%%%%%%%%%%%%%%%%%%%%%%%%%%%%%%%%%%%%%%%%%%%%%%%%%%%%%%%%%%%
%% abstrac-en.tex
%% NOVA thesis document file
%%
%% Abstract in English([^%]*)
%%%%%%%%%%%%%%%%%%%%%%%%%%%%%%%%%%%%%%%%%%%%%%%%%%%%%%%%%%%%%%%%%%%%

\typeout{NT FILE abstrac-en.tex}%

In contrast to monolithic system designs, microservice architectures provide greater scalability,
availability, and delivery by separating the elements of a large project into independent
entities linked through a network of services.
Because services are tied to one another via their interfaces, they can only evolve separately if their contracts remain consistent.
There is a scarcity of mechanisms for safely evolving and discontinuing functionalities of services.

In monolithic system design's, changing the definition of an element can be accomplished
quickly with the aid of developer tools, such as IDE refactoring toolkits.
In distributed systems there is a lack of tools comparable to those used in centralized systems, developers are left
with the burden of manually tracking down and resolving problems caused by uncontrolled updates.
To ensure that microservices are working properly the general approach is to validate their behaviour through empirical tests.

This thesis aims to supplement the conventional approach by providing mechanisms that
support the automatic validation of deployment operations, and the evolution of microservice interfaces.
It´s presented a microservice management system that verifies the safety of modifications to
service interfaces and that enables the evolution of service contracts without impacting consumer services.
The system use runtime-generated proxies, that dynamically convert the data sent between services to the format
expected by static code, and thereby relieving the developer of the need to manually adapt
existing services.

% Palavras-chave do resumo em Inglês
\begin{keywords}
    microservices, software evolution, service compatibility, API, interface adaptation
\end{keywords} 
