%!TEX root = ../template.tex
%%%%%%%%%%%%%%%%%%%%%%%%%%%%%%%%%%%%%%%%%%%%%%%%%%%%%%%%%%%%%%%%%%%%
%% chapter4.tex
%% NOVA thesis document file
%%
%% Chapter with lots of dummy text
%%%%%%%%%%%%%%%%%%%%%%%%%%%%%%%%%%%%%%%%%%%%%%%%%%%%%%%%%%%%%%%%%%%%

\typeout{NT FILE chapter4.tex}%

\chapter{Implementation}
\label{cha:implementation}

In this chapter, we present the implementation of our system.
We begin by offering an overview of system's architecture,
before delving into the implementation of each module in depth.

\section{Architecture} % (fold)
\label{sec:architecture}


The system was built using a modular approach;
each module is entirely self-contained, making it simple to expand or replace system functionality.
The system architecture is split into two distinct areas: the runtime, and the pipeline.
The runtime architecture comprises all essential components to the system liveness,
whereas the pipeline architecture covers all components needed during the system's provisioning and development.

\subsection{Runtime Architecture} % (fold)
\label{sec:runtime_architecture}

The runtime environment is supported by Kubernetes,
a simplified configuration with three microservices is depicted in Figure ~\ref{fig:runtime},
in this example, we have one micro-service that has undergone three major API changes the accounts micro-service, and
two micro-services that rely on it, the inventory micro-service that was updated to account for the changes, and the shipping micro-service which is two versions behind.

In Kubernetes a microservice is internally accessible through the Service abstraction;
a Service defines a policy for accessing a logical set of Pods via a load balancer with a
unique IP address that can be discovered thorough an environment variable, or an DNS name.
The containers of a pod are selected in services policies by attributing names to their exposed ports and assigning the same name on the target port of the service policy.
In adapter containers we expose one port for each of the supported versions and name each port with the respective version.

We define one service for each major version that is still being consumed, as shown in Figure ~\ref{fig:runtime},
older versions are supported by the adapter, while the most up to date version is directly supported by the microservice app.
A consumer accesses a producer app via the service that haves the same version as the one that it's expected in the static code.
When a microservice app is upgraded via a rolling update \footnote{A rolling update gradually replaces old replicas with new replicas},
the services never become unavailable, they point either to the adapter containers or the application containers, depending on whether the affected pods finished the upgrade process,
whereas in the traditional approach all consumers would need to be upgraded in tandem with consumed services in order to avoid downtime.

The only novel module introduced in the runtime environment is the proxy adapter;
the forthcoming modules are either developer tools or are invoked on stages of the DevOps pipeline.

\begin{figure}[htbp]
    \centering
    \includegraphics[height=2.4in]{runtime}
    \caption{Example of a runtime configuration}
    \label{fig:runtime}
\end{figure}

\subsection{DevOps Pipeline Architecture} % (fold)
\label{sec:devops_pipeline_architecture}

A proof-of-concept deployment pipeline as developed, its architecture is depicted in Figure ~\ref{fig:pipeline}.
The nodes on the left side of the graph represent the pipeline;
development stages are represented in green, while operational stages are represented in red.
The other nodes on the graph's reflect the resources used in each stage.
External resources are depicted in blue, while the novel modules developed in this thesis are depicted in black.
Two novel stages were introduced in the development phase: the writing of service contracts, and the specification of contract evolutions.
These two stages are skipped if a service only underwent changes that did not affect its API or dependencies.
We also introduced two novel stages in the development phase:
the verification of the service compatibility with the rest of the system, and the generation of adapter component.

The pipeline was developed using the Jenkins pipeline suite, each of the novel stages is supported by a distinct Java application that is pulled from version control
and executed directly on the pipeline, whereas the other stages are supported natively by the Jenkins suite toolkit.
The modules that enable each novel stage will be discussed in detail in the following section.

\begin{figure}[htbp]
    \centering
    \includegraphics[height=9in]{DevOps}
    \caption{DevOps Pipeline}
    \label{fig:pipeline}
\end{figure}

\newpage

\section{Contract Representation} % (fold)
\label{sec:contract_representation}

The OpenAPI specification is the most
widely adopted WADL for HTTP services, instead of designing yet another WADL, it was
adopted the OpenAPI specification to support the needs of the implementation.
Designing a WADL tailored for this problem would be counterproductive because teams rely on other tools,
and supporting an additional WADL would require the development of an extensive converter tool
that would need to be constantly maintained in order to support changes in new versions of the OpenApi and the new WADL specification's.

\paragraph{}

To represent records in messages, it was chosen the JSON format, because of its simplicity and widespread adoption.
There are more performant formats that have a binary representation,
however such formats are usually not supported by front-end frameworks and for the purpose of this prototype
we are not interested in measuring the performance of different object serialization protocols, since
there are already sources that provide comparisons in this regard between XML, JSON, Protobuf, Thrift and other popular formats \cite{serializationBenchmark}.

\paragraph{}

The schema of JSON records must also be described,
the OpenAPI specification supports the description of records
in conjunction with the signature of HTTP endpoints.
Other JSON schema specification languages, such as the JSON-Schema project
exist, but using languages other than OpenApi would necessitate decoupling the specification of schemas from the specification
of HTTP endpoints, which would require additional effort on the part of the developing team to maintain the cross-references
between each specification manifest, and their versions.

\paragraph{}

A snippet of a "Pet Store" OpenApi contract can be seen in Figure ~\ref{fig:open_contract}.
In this example we can view the specification of a JSON schema for a Pet data object,
as well the specification for an HTTP endpoint.
OpenApi allows properties to be defined robustly with specification of not only their type but also their format, as shown in line 31.

\paragraph{}

The types and formats of properties are standardized between object schemas and http parameters;
properties have the following data types: string, number, integer, boolean, array, and object;
while property formats are free-form, there are established conventions for the most common types.

\begin{figure}[htbp]
    \centering
    \includegraphics[height=6in]{openContract}
    \caption{OpenAPI Contract Example}
    \label{fig:open_contract}
\end{figure}

\newpage

The specification of the dependencies of a service was done using Helm \textbf{[TODO: reference]}.
Helm is package manager for Kubernetes that allows developers to more easily package, configure, and deploy applications and services onto Kubernetes clusters.
A chart is a Helm package, composed of a collection of files that describe a related set of Kubernetes resources.
The most significant components of a Chart are the:
\begin{itemize}
    \item The Chart.yaml, that allows the definition of the Chart dependencies and other meta-information about the application such as its name and version.
    \item The Template's directory, that contains incomplete Kubernetes manifests "Templates" such as Services, Deployments, DaemonSets, Namespaces and so on.
    \item The Values.yaml, that contains key-value pairs used to complete the definition of each template, this values can be overridden later when the chart is deployed, the file is used to define the default values.
\end{itemize}
A single chart might be used to deploy something simple, like a single kubernetes pod, or something complex, like a complete web app stack with HTTP servers, databases, caches, and so on.
The typical approach is to define an individual chart for each distinct microservice, so that they can be upgraded individually.
If a single chart is used to deploy multiple services it is no longer possible to represent accurately the dependencies of each service
in the chart definition, because the chart syntax doesn't allow the individual definition of the dependencies of each service, it only allows the declaration of the dependencies of the deployment as a whole.

\paragraph{}

An example of the definition of service dependencies can be seen in Figure ~\ref{fig:serviceDepedencies}.
The dependencies of a service are mapped indirectly, each dependency point to a Helm chart that is implicitly associated with one service.

\begin{figure}[htbp]
    \centering
    \includegraphics[height=2in]{serviceDepedencies}
    \caption{Helm Chart Dependencies Definition Example}
    \label{fig:serviceDepedencies}
\end{figure}

\newpage

The declaration of resources consumed by a service is defined in chart template files.
For each deployment template it was defined the resources consumed by its containers by
setting the minimum amount of cpu and memory resources (represented on the requests key) and by setting a limit on these resources (represented on the limits key).
An example of the definition of a service resources can be seen in Figure ~\ref{fig:serviceResources}.

\begin{figure}[htbp]
    \centering
    \includegraphics[height=3in]{serviceResources}
    \caption{Helm Template Resources Definition Example}
    \label{fig:serviceResources}
\end{figure}

\section{Contract Interpretation} % (fold)
\label{sec:contract_interpretation}

The interpretation of contract specification is done in a dedicated module named "Parser".
This module is meant to be used as Java library by other modules,
it interprets contracts written in OpenApi format and provides a read-only Java interface
that is agnostic to the original specification format.

\paragraph{}

The implementation of the Parser module uses the swagger-parser project \textbf{[TODO: reference]} to convert OpenApi contracts into a Java POJO representation, that is accessible via a simplified read-only interface.
All the forthcoming modules operate under this interface not under the OpenApi format.
Although it was adopted the OpenApi format the solution is not dependent on it, this module provides a decoupling point.
This enables the specification of different contract to be written in different or even mixed WADL,
as long as the Parser module is extended to support the new WADL's.
The interpretation of service resources and dependencies written in Helm charts is also supported by the Parser module under a separate interface.

\newpage

\section{Evolution Representation} % (fold)
\label{sec:evolution_representation}

The evolution of contracts is represented in a custom description language.
The description only maps the details about the changed procedures of a service,
it is not necessary to map the evolutions in service dependencies and resource requirements because they can be inferred by comparing the last two contract versions.
The description language is hierarchical and uses array elements.
The following Figure ~\ref{fig:evolution_yal} represents its structure, and Figure ~\ref{fig:evolution_example} gives an example.
Array elements are depicted by the [.] notation.

\paragraph{}

As previously mentioned in the design chapter, resolutions can have one of three distinct types
DefaultValues, Link or Function.
The type of resolution is represented in the first segment of a resolution, while the second segment represents the resolution itself.
Link resolutions have the following format "link={Parameter Location}|{Parameter Name}", the parameter of the prior contract must be of the same type and format.
Default value resolutions have the following format "default=#Value", the value that must be of the same type and format.

\paragraph{}

Function resolutions are written as one line Java expressions that can use parameters of the previous contract, the
parameters are referenced by their location and name (eg. function= {{json|firstName}} + {{json|lastName}} ).
These references are replaced in the adapter implementation by their real value.

\begin{figure}[htbp]
    \centering
    \includegraphics[height=4.2in]{EvoExample}
    \caption{Evolution Specification Example}
    \label{fig:evolution_example}
\end{figure}

\begin{figure}[htbp]
    \centering
    \includegraphics[height=5in]{evolutionYal}
    \caption{Evolution Specification Structure}
    \label{fig:evolution_yal}
\end{figure}

\newpage

\begin{figure}[htbp]
    \centering
    \includegraphics[height=2.5in]{editor2}
    \caption{Evolution editor, procedure mappings}
    \label{fig:editor1}
\end{figure}

\begin{figure}[htbp]
    \centering
    \includegraphics[height=4.5in]{editor1}
    \caption{Evolution editor, parameter mappings}
    \label{fig:editor2}
\end{figure}

\newpage

To minimize the documentation effort it was developed a GUI editor for the facilitates the specification of evolutions.
Figures ~\ref{fig:editor1} and ~\ref{fig:editor2} show the editor.
The GUI editor was implemented as an IDE Plugin, the current implementation supports the Intellij integrated development environment.

\paragraph{}

The editor compares two contract versions,
and automatically maps all the elements that haven't changed from one version to the other, the
developer is left with the task of explicitly mapping the remaining elements.

\paragraph{}

The editor starts by requesting the developer to map endpoints in the new contract to endpoints in the old contract,
afterwards the developer must edit each of the endpoint messages and specify a resolution for all the parameters that changed
in a contract, unresolved parameters are highlighted in red.
The right column holds the resolutions that were already mapped and uppermost left column allows the developer to write rules
for each unresolved parameter.
The developer can select leaf parameters in the hierarchy, or can choose intermediate properties.
If he chooses an intermediate property and inputs resolution rule all child parameters are also resolved.
After all parameters are resolved the developer submits the resolutions, and
the editor outputs an evolution specification file with the same description language presented in the beginning of this section.

\section{Compatiblity Verification} % (fold)
\label{sec:compatiblity_verification}

The compatibility verification process is done by a dedicated module named "Verifier".
This module was implemented in java as command-line application.
The module verifies if the provided evolution specification is valid.
It starts by checking if all the new contract procedures, messages and parameters are present in the specification,
then it verifies if all the resolutions for each parameter are valid.
Resolutions can be invalid due to four reasons:

\begin{itemize}
    \setlength\itemsep{0em}
    \item Invalid syntax in resolution expressions
    \item References parameters that don't exist in the previous contract
    \item Uses functions that are not supported
    \item Uses resolutions that have a different type and format then the parameter
\end{itemize}

The present implementation of the Verifier module does not test whether the cluster has adequate computational resources to provision the service or whether all service dependencies are alive and reachable.
These functionalities were not implemented since the Helm deployment process already performs these verifications prior to the deployment of a service.
Nonetheless, including these verifications in the Verifier module would be useful in order for the solution to be decoupled from Helm.

\textbf{TODO: pseudo codigo}

\newpage

\section{Adapter Generation} % (fold)
\label{sec:adapter_generation}

The generation of the proxy adapter code is done in a dedicated module named "Generator".
The adapter is generated by populating a template implementation with the information provided in evolution specs.
The adapter must be able to support all prior versions that are still in use, and adapt to the last version, in other words,
a single adapter is used to support the adaptation of multiple versions to the current version.

The evolution specifications only provide information on how to adapt messages from contiguous versions (eg. v1->v2).
The information necessary to adapt messages from discontinuous versions is obtained by chaining the evolution specifications and
merging their resolutions (e.g merge[v1->v2, v2->v3] = v1->v3).

This module begins by querying the active versions of the target service with the Kubectl command-line tool.
Kubectl is a tool that interfaces with Kubernetes clusters and can be used to inspect and manage cluster resources.
After determining the active service versions, this module retrieves all evolution specifications up to the oldest version that is still active,
and then combines them so that all specifications link to the current version. Figure ~\ref{fig:merge} exemplifies the merge process.

\begin{figure}[htbp]
    \centering
    \includegraphics[height=3in]{merge}
    \caption{Evolution Specification Merge Process}
    \label{fig:merge}
\end{figure}

A single proxy adapter supports the adaptation of messages for multiple versions of the same service contract.
The implementation exposes each version of the contract adaptation procedures in a distinct port.
A Kubernetes service is created for accessing each of the adapter ports.
It is also possible to version procedures with header parameters,
but the generated adapter code must be more complex because a single endpoint must be able to adapt messages from multiple versions.

\section{Adapter} % (fold)
\label{sec:adapter}

The adaption approach consists on: for each endpoint the
adapter re-constructs the entire request from zero and populates the parameter resolutions with the information of the original request.
The "Function" resolutions are written as one-line java expressions.
The resolutions are injected directly into the template implementation of the adapter, and the parameter references in the resolution are then replaced with the parameters themselves.

\textbf{TODO: pseudo codigo}

\paragraph{}

The adapter's current template implementation does not support HTTPS, however this can be added.
Because TLS is mandatory in HTTP 1.2, the implementation only supports HTTP 1.1.

\paragraph{}

The adapter was built with Spring Boot and an Apache Tomcat/9.0.65 http server, the base image of the adapter docker container is an "openjdk:11-nanoserver".
In order to prevent the JVM from calling garbage collector frequently and reallocating the heap memory while Tomcat is trying to serve requests,
the JVM has started with a higher maximum heap memory, and the initial heap memory
size was set to the same value as its maximum memory size.
The maximum number of threads for Tomcat was set to 2000, in order to support a higher load of requests.

