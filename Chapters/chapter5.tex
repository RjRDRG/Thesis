%!TEX root = ../template.tex
%%%%%%%%%%%%%%%%%%%%%%%%%%%%%%%%%%%%%%%%%%%%%%%%%%%%%%%%%%%%%%%%%%%%
%% chapter4.tex
%% NOVA thesis document file
%%
%% Chapter with lots of dummy text
%%%%%%%%%%%%%%%%%%%%%%%%%%%%%%%%%%%%%%%%%%%%%%%%%%%%%%%%%%%%%%%%%%%%

\typeout{NT FILE chapter4.tex}%

\chapter{Background}
\label{cha:background}

\section{API} % (fold)
\label{sec:api}

An application programming interface (API) allows software modules (eg. products, services, libraries)
to easily communicate with one another by abstracting their underlying implementations and only exposing the relevant objects and operations.

The operations that make up a API are sometimes known as subroutines, methods, requests, or endpoints.
An API specification defines these operations, meaning that it explains how to use or implement them.
APIs are sometimes thought of as contracts, with documentation that represents an agreement between parties:
If party 1 sends a remote request structured a particular way, this is how party 2’s software will respond.

APIs provide flexibility and make the design and administration of sofware  easier.

\section{Agile Development} % (fold)
\label{sec:agile_development}

Plan-driven engineering requires us to create a predictive plan before beginning development.
The plan outlines the project's people, resources, and timelines.
Software design is likewise done ahead of time, with implementation anticipated to follow design.

Agile plans serve as a foundation for managing change.
Agile teams plan just as meticulously as conventional teams, but their plans are regularly revised to reflect what they learn during development.

\section{Microservices} % (fold)
\label{sec:microservices}

\section{Microservice Communication Strategies} % (fold)
\label{sec:microservice_communication_strategies}
 web api
 rpc
 event driven

\section{Microservice Orchestration Tools} % (fold)
\label{sec:microservices}

\section{Type System} % (fold)
\label{sec:type_system}
