%!TEX root = ../template.tex
%%%%%%%%%%%%%%%%%%%%%%%%%%%%%%%%%%%%%%%%%%%%%%%%%%%%%%%%%%%%%%%%%%%%
%% chapter4.tex
%% NOVA thesis document file
%%
%% Chapter with lots of dummy text
%%%%%%%%%%%%%%%%%%%%%%%%%%%%%%%%%%%%%%%%%%%%%%%%%%%%%%%%%%%%%%%%%%%%

\typeout{NT FILE chapter4.tex}%

\chapter{Proposal}
\label{cha:proposal}

The versions of service API are written in standard specification language such as OpenAPI;

The service API is converted to a native specification language that is agnostic from the way data is transmitted (Http, Event stream, etc..)
As a result, the system is loosely coupled to the transmission method, and if a new transmission methodology is widely adopted, the system does not need to be remade,
but only extended to support the extraction and injection of parameters in the new methodology.

The versions of service API are collected in a service registry;

When a new version is submitted for deployment the registry is consulted in order to obtain the set of all active versions of the same service;

For each pair of versions (new, old) a compatibility specification file is produced;

//TODO figure with compatibility specification

In this file all functions and parameters are associated with a key;

Records are expanded/exploded and each of its parameters is associated with a key, the record itself is also associated with a key;

This file contains the associations between the old version function/parameter keys, and the new version parameters keys (eg. old:k0 -> new:k1; old:k1->new:k5);

This file is audited by a programmer before being used for the generation of the adapter proxy code
In case of incorrect mapping between parameter types the programmer can fix manually.

Some ambiguous evolutions can only be solved with human intervention and knowledge of the domain;
The system notifies the programmer if such a case occurs and request intervention for its resolution.

Examples:
renaming two fields of the same type;
inserting a new field and removing another of the same type, is indistinguishable from renaming a field;

Default values can also be supplied inside compatibility specification in order to de-clutter service API or implementations;

More complex signature changes (such as the change of parameter unit from celsius to fahrenheit, the change of date format, the partition of name parameter into surname and first name)
can be accommodated in the combinability specification with the use of functions (eg. extractFirstName(old:k0) -> new:k1  extractSurname(old:k0) -> new:k1)

This functions can be made available through static libraries, or more interestingly, the programmer could write their implementation in the compatibility specification file
and afterwards they would be integrated in adapter proxy coded before compilation;


\section{Adaptation Methodology} % (fold)
\label{sec:adaptation_methodology}

The adaptation methodology inputs two communicating BPEL processes, C
and S, whose interaction may lock, and it builds (if possible) a BPEL process
adapter A, which allows the two processes to successfully interoperate. The four
adaptation phases are: (1.) Service Translation. This phase is in charge of
translating the BPEL descriptions of C and S into corresponding YAWL work-
flows [7]. (2.) Adapter Generation. This phase builds the YAWL workflow of
A from the workflows of C and S. It first generates the Service Execution Trees
(SETs) of C with respect to S (SET (CS)), as well as of S with respect to C
(SET (SC)), followed by the generation of the SETs of their duals (SET (CS)
and SET (SC)). Informally, when a service X outputs a message m, a dual of
X is a service that inputs m, and vice-versa. Next, SET (A) is obtained by
suitably merging SET (CS) and SET (SC). Finally, the YAWL workflow of A
is derived from SET (A). (3.) Lock Analysis. This phase verifies whether the
YAWL-based aggregation [4,6] of C, A, and S locks. If it does, we consider that
the adaptation has failed. Otherwise, we consider that the adaptation is suc-
cessful. (4.) Adapter Deployment. If the adaptation is successful, this phase
deploys the YAWL workflow of A as a BPEL process, which can be used as a
service-in-the-middle between C and S.

Service Translation
In [7] we present a methodology for translating BPEL processes into YAWL
workflows. Its main strengths are that (1) it defines YAWL patterns for all
BPEL activities, (2) it provides a compositional approach to construct structured
patterns from suitably interconnecting other patterns, and (3) it handles events,
faults and (explicit) compensation.
On the one hand, the pattern of each BPEL basic activity (with the excep-
tion of assign and compensate) is obtained by suitably instantiating the Basic
Pattern Template (BPT). The BPT is a template of YAWL tasks, which serves
both for identifying the translated activity (through an Activity Specific Task,
or AST for short), as well as the control-logic of executing or skipping the activ-
ity. On the other hand, the pattern of each BPEL structured activity (together
with assign, compensate, and process) is obtained from the Structured Pattern
Template (SPT) template. The SPT consists of a Begin (logically marking the
initiation of the structured activity) and of an End pattern (logically marking
the termination of the structured activity), as well as a pattern template (BPT
or SPT) for each child activity. Furthermore, the Scope and Process patterns add
SPTs for handling exceptional behaviour. Each pattern inputs and outputs at
most three types of control-flow links, called green, blue, and red lines. The green
lines serve for translating the structural dependencies among BPEL activities.
The blue lines are used for translating the BPEL synchronisation links, and the
red lines are necessary for implementing the fault handling mechanism. As space
limitations do not allow us to go into further details, please see [7] for more
(in-depth) details on the BPEL2YAWL translator.
The YAWL workflows of the CC and ME services of our example can be
seen in Figure 1.2 In the workflow of ME, Begin(Process) and End(Process),
logically mark the initiation and the termination, respectively, of the BPEL pro-
cess. The process activity, a sequence leads to generating the Begin(Sequence) as
well as the End(Sequence) tasks. The first activity in the sequence is a receive,
which gives the Receive task. Furthermore, the rest of the activities are trans-
lated correspondingly. (The numbers inside some of the task labels are used for
disambiguation purposes only.) Note however the translation of the BPEL pick.
The Begin(Pick) task contains the branch selection logic (basically a deferred
choice construct [13]), and it outputs two tokens3. One leads to executing the
chosen branch, while the second leads to skipping the other branch (so as to
achieve the dead-path-elimination).
The workflow of CC is built in a similar manner. However, the compos-
ite tasks representing the invoke ValidateLocation and invoke Explore activities
output either green tokens, if the invocations succeed, or red tokens, if the in-
vocations fail (i.e., faults are being raised). In the former case, the execution of
the workflow continues normally, and the green output of End(Sequence) leads
to skipping the tasks inside the Begin(FaultHandler) → End(FaultHandler) zone
(so as to achieve the dead-path-elimination). In the latter case, the execution of
the faulty invocation is (immediately) followed by the execution of the tasks in
the fault handling zone.