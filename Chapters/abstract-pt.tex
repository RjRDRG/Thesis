%!TEX root = ../template.tex
%%%%%%%%%%%%%%%%%%%%%%%%%%%%%%%%%%%%%%%%%%%%%%%%%%%%%%%%%%%%%%%%%%%%
%% abstrac-pt.tex
%% NOVA thesis document file
%%
%% Abstract in Portuguese
%%%%%%%%%%%%%%%%%%%%%%%%%%%%%%%%%%%%%%%%%%%%%%%%%%%%%%%%%%%%%%%%%%%%

\typeout{NT FILE abstrac-pt.tex}%

Em contraste com sistemas tradicionais monoliticos, as arquiteturas de microsserviços
permitem grande escalabilidade, disponibilidade e capacidade de entrega, separando os
elementos de um grande projeto em entidades independentes ligadas através de uma rede
serviços. Como os serviços estão ligados uns aos outros através das suas interfaces, só
podem evoluir separadamente se os seus contratos se mantiverem consistentes. No
entanto, existe uma escassez de mecanismos para evoluir e descontinuar as
funcionalidades dos serviços em segurança.

Nos sistemas tradicionais monoliticos, a alteração da definição de um elemento pode ser
realizada rapidamente com a ajuda de ferramentas automatizadas (tais como kits de
ferramentas de refactoring IDE). Em sistemas distribuídos, existe falta de ferramentas
comparáveis às utilizadas em sistemas centralizados, os programadores ficam normalmente
sobrecarregados com a resolução manual de problemas causados por atualizações e pela
validação do correcto funcionamento do sistema através de testes empíricos.

O trabalho desenvolvido nesta tese procura complementar a abordagem convencional,
fornecendo mecanismos que suportam a validação das operações de deployment. É
apresentado um sistema de gestão de microsserviços que verifica a segurança das
modificações das interfaces de serviço e a evolução dos contratos. A abordagem utiliza proxies,
que convertem dinamicamente os dados enviados entre serviços ao formato esperado pelo
código de serviço estático, minimizando a intervenção manual do programador.

\begin{keywords}
    microsserviços, evolução de software, compatibilidade de serviços
\end{keywords}
% to add an extra black line
