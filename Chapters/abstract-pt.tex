%!TEX root = ../template.tex
%%%%%%%%%%%%%%%%%%%%%%%%%%%%%%%%%%%%%%%%%%%%%%%%%%%%%%%%%%%%%%%%%%%%
%% abstrac-pt.tex
%% NOVA thesis document file
%%
%% Abstract in Portuguese
%%%%%%%%%%%%%%%%%%%%%%%%%%%%%%%%%%%%%%%%%%%%%%%%%%%%%%%%%%%%%%%%%%%%

\typeout{NT FILE abstrac-pt.tex}%

Em contraste com sistemas tradicionais, as arquiteturas de microsserviços permitem grande
escalabilidade, disponibilidade e entrega separando os elementos de um grande projeto em entidades independentes ligadas através de uma rede (serviços), porém não oferecem
mecanismos de segurança para evolução e descontinuação de funcionalidades fornecidas pelos serviços.

Alterar a definição de um elemento em um programa convencional pode ser feito rapidamente com o auxílio de ferramentas automatizadas.
Os sistemas distribuídos não possuem ferramentas comparáveis às utilizadas em sistemas centralizados.
Na ausência de ferramentas que amenizem este problema,
os desenvolvedores tem que rastrear e resolver manualmente os problemas causados por atualizações não controladas.

Enquanto que as abordagens tradicionais garantem que os microsserviços obcedem a sua especificação, atráves da validação do seu comportamento por meio de testes empíricos,
a nossa proposta procura complementar a abordagem convencional fornecendo mecanismos que suportam a validação das operações de deployment e a evolução das interfaces de microsserviços.

Apresentamos um sistema de administração de microsserviços que verifica a segurança de modificações nas interfaces de serviços
e permite a evolução de interfaces compatíveis por meio de componentes proxy gerados em tempo de execução
que adaptam dinamicamente os dados trocados entre serviços ao formato esperado pelo código de serviço estático,
aliviando assim o developer da necessidade de adaptar manualmente serviços novos ou existentes.

\begin{keywords}
    microsserviços, evolução de software, compatibilidade de serviços
\end{keywords}
% to add an extra black line
