%!TEX root = ../template.tex
%%%%%%%%%%%%%%%%%%%%%%%%%%%%%%%%%%%%%%%%%%%%%%%%%%%%%%%%%%%%%%%%%%%%
%% chapter4.tex
%% NOVA thesis document file
%%
%% Chapter with lots of dummy text
%%%%%%%%%%%%%%%%%%%%%%%%%%%%%%%%%%%%%%%%%%%%%%%%%%%%%%%%%%%%%%%%%%%%

\typeout{NT FILE chapter4.tex}%

\chapter{Introduction}
\label{cha:introduction}

Architecture is the most important factor that influences the capabilities and constraints of a system's evolution throughout its production.
In software engineering, architecture is concerned with the conciliation of various architectural elements with the objective of satisfying functional and non-functional requirements of a system.

That is, a software architecture can be seen as a set of architectural elements that are divided into three classes: processing elements, data elements, and connecting elements [Perry and Wolf 1992].
The processing elements are those that perform transformations on the data elements;
The data elements are those that contain the information that is used and transformed;
The connecting elements (which at times may be either processing or data elements, or both) are the glue that holds the different pieces of the architecture together.
(eg. procedure calls, shared data, messages \ldots)
It is the connecting elements that are frequently used differentiate one architecture from another,
because among the other elements, they have the greatest impact on non-functional requirements, such as the system's ability to evolve, security, scalability and so on.

The field of software architecture dates back to the 1960s and 1970s, but a solid foundation on the topic was only established in 1992 by Perry and Wolf.
Since then, it has generated a large community of researchers studying the notion and its practical applications,
allowing the concepts to be widely adopted by both industry and academia.
Bosch's work [Jan Bosch. Software architecture: The next step. In Software architecture] provides a good overview of the current research in software engineering and architecture.

Over the past several decades, software architecture has been thoroughly studied, and as a result software engineers have come up with different ways
to compose systems that provide broad functionality and satisfy a wide range of requirements.

A typical example of a monolithic architectural design pattern is the Model-View-controller (MVC).
Although the MVC pattern encourages separate layers, which can improve software maintainability,
such structures are notorious for their lack of agility and scalability due to their monolithic nature.
Creating instances of massive monoliths is time-consuming and inefficient;
even minor system updates require the complete redeployment of the application, severely limiting a system's ability to evolve.

Service-Oriented Computing (SOC) is an emerging paradigm for distributed computing.
In SOC, a program known as a service provides functionality to other components via network message passing.
Messages are exchanged using service interfaces, which are implementation-independent.
The first generation of service-oriented architectures (SOA) set imposing and ambiguous criteria for services, which hampered its adoption.
Microservices are the second iteration of the SOC idea;
their aim was to reduce superfluous levels of complexity so that developers can focus on building simple services that efficiently fulfill a specific task.

Some advantages of service orientation over monoliths are as follows:
\begin{itemize}
    \item Functional independence - Each service is operationally independent to others, the only way they communicate is through their published interfaces.
    \item Component Scalability - When a single component becomes overloaded, additional instances of the component can be deployed separately.
    \item Distributed development - By agreeing on the interfaces of the distributed system, distinct teams can develop partitions of the system in parallel.
\end{itemize}

These advantages boost the system's ability to evolve by allowing it to be disassembled into subsystems composed primarily of loosely coupled services.

Subsystems can fully evolve independently of one another, as long they don't share services/components.
Because services are tied to one another via their interfaces, they can only evolve separately if their contracts remain consolidated.
It is often advantageous to have the capacity to evolve services contracts independently,
particularly in agile workflows and the early phases of development, where contracts are prone to frequent changes.

\section{Problem Statement} % (fold)
\label{sec:problem_statement}

The advantages of employing microservices-based architectures are frequently outweighed by the lack of mechanisms for assessing the safety of service contract updates.

In a monolithic application, interactions between different system components are conducted through function calls.
In a distributed system, however, each individual component has to communicate with other components across the network.
In a regular program, refactoring a function definition can be done rapidly with the aid of an IDE.
Distributed systems do not have equivalent tools to detect mismatches between endpoints and their consumers.
Developers are left with the burden of tracking down manually system's dependencies and refactor them in all consumers and producers.

With a rising number of separate services and their interactions, contract administration and service integration become progressively more challenging.
Software engineering guidelines assume a scenario of frequent service implementation changes and few interface or contract modifications;
in the context of iterative approaches to software development, such as agile, contracts are typically modified as frequently as implementation details.

Maintaining the integrity of microservice architectures is a daunting task that only the most diligent teams can handle.

\paragraph{The ramifications of changing a contract are discovered after the fact.}

Syntactic or semantic changes in service contracts are usually silent because services evolve independently and contracts are rarely formally specified or even documented.
Without explicit contract verification at deployment time, the only symptoms that the system is broken are runtime errors or unexpected behaviors.
It is paramount to have contract management methods and a common contract specification in place, as this allows the safety of service-based architectures to be checked a priori.

\paragraph{The safety of a deployment operation is unknown.}

Established platforms such as Kubernetes cannot ensure the safety of deployment operations since they rely primarily on service level meta-information (such as version IDs) to manage deployments.
To assure the safety of deployment tasks, a new model for such environments is required,
one that uses generic meta-information from previously deployed services, adopts expanded type-based contracts and employs a compatibility relation on service contracts.

\paragraph{A contract change, entails the redeployment/downtime of its consumers.}

Modifications to type definitions in a producer module have immediate
impact on the whole system by requiring downtime on all consumer modules.
Deployments of “compatible” modules should not disrupt the system soundness by imposing the redeployment of services;
instead, one service should be able to be replaced while the remainder of the system remains operational.

\section{Contributions} % (fold)
\label{sec:contributions}

We present a microservice management system that aims to solve the problems stated above.

The first and second problems will be tackled by defining and adopting an API specification language, as well as implementing a schema
registry that offers a pre-flight safety check procedure that verifies whether two service contracts are compatible before
deployment. The compatibility mechanism, is akin to typechecking procedures in compiled languages, and is applied upon a
pair of service contracts, which are typically two distinct versions of the same service. The aforementioned mechanism preserves
a distributed system's correct behavior by rejecting additions or modifications that would threaten it and by informing the programmer
of all the repercussions of the deployment, such as consumer services that are outdated.

To solve the third problem the evolution of interfaces is supported at runtime by generated proxy components that dynamically
adapt data exchanged between services to match the services contracts.
The proxy adapter is capable intercepting and evaluating all the outgoing and incoming TCP requests to determine whether
they should be left unchanged or how they should be transformed using an adaptation protocol that extends the compatibility mechanism discussed above.
We allow for addition, removal, renaming and type migration of data fields of a producer module without breaking or upgrading consumer services.
More complex changes in data fields, such as altering the format of a date, will be permitted through the use of user-defined adaption functions.
We also support the modifications on the signatures of APIs, such as changing a path parameter in a REST endpoint to a query parameter.
The translation specification between two compatible contracts can be audited and modified by a programmer before it is employed by an adapter.

Our work will make the following contributions: (details will be presented on chapter 4)

\begin{itemize}
    \item A compatibility relation on service contracts that determines whether or not messages may be exchanged without data loss.
    \item Implementation of a robust type-directed adaptation protocol for REST APIs.
    \item Definition of a core language for the specification of contract compatibility corrections.
    \item Implementation of a service registry tasked with holding service contracts, tracking consumers of service versions,
    and providing a lightweight preflight safety check procedure for deployment/un-deployment operations.
    \item Implementation of a benchmark platform to compare the solution to existing ones.
\end{itemize}

\section{Document Scructure} % (fold)
\label{sec:document_structure}