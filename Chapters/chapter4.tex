%!TEX root = ../template.tex
%%%%%%%%%%%%%%%%%%%%%%%%%%%%%%%%%%%%%%%%%%%%%%%%%%%%%%%%%%%%%%%%%%%%
%% chapter4.tex
%% NOVA thesis document file
%%
%% Chapter with lots of dummy text
%%%%%%%%%%%%%%%%%%%%%%%%%%%%%%%%%%%%%%%%%%%%%%%%%%%%%%%%%%%%%%%%%%%%

\typeout{NT FILE chapter4.tex}%

\chapter{Introduction}
\label{cha:introduction}

Architecture is the most important factor that influences the capabilities and constraints of a system's evolution throughout its lifecycle \cite{Breivold2012}.
In software engineering, the subject of architecture is the conciliation of various architectural systems or components
with the objective of satisfying a system's functional and non-functional requirements.

That is, a software architecture can be seen as a set of architectural elements that are divided into three classes: processing elements, data elements, and connecting elements \cite{architecture}.
The data elements contain the state of system;
The processing elements transform the data elements.
The connecting elements, which may be either processing or data elements, or both, are the glue that keeps the architecture together.
(e.g.\ procedure calls, shared data, messages)
The connecting elements are frequently used to differentiate one architecture from another,
because among the other elements, they have the greatest impact on non-functional requirements, such as the system's ability to evolve, security, scalability, and so on.

\paragraph{}

The field of software architecture traces back to 1960s, but it wasn't until 1992 that \citeauthor{architecture} \cite{architecture} created a firm basis on the subject.
Software architecture has been extensively studied over the last several decades, and as a result,
software engineers have developed novel methods for composing systems that offer broad functionality and fit a diverse set of criteria.
Bosch's work \cite{3, 4} overviews software architectural research.

\paragraph{}

A typical example of a monolithic architectural design pattern is the Model-View-controller (MVC) \cite{mvc, microservices}.
Although the MVC pattern encourages separate layers, which can improve software maintainability,
such structures are notorious for their lack of agility and scalability due to their monolithic nature.
Creating instances of massive monoliths is time-consuming and inefficient;
even minor system updates require the complete redeployment of the application, severely limiting a system's ability to evolve.

\paragraph{}

Service-Oriented Computing (SOC) \cite{6} is a paradigm for distributed computing.
In SOC, a program known as a service provides functionality to other components via network message passing.
Messages are exchanged using service interfaces, which are implementation-independent.
The first generation of service-oriented architectures (SOA) \cite{7} set imposing and ambiguous criteria for services, which hampered its adoption.
Microservices \cite{microservices} are a modern iteration of the SOC idea;
their aim is to reduce excessive layers of complexity so that developers can focus on building simple services that efficiently fulfill a specific task.

\paragraph{}

Some advantages of service orientation over monoliths are as follows:
\begin{itemize}
    \item Functional independence - Each service is operationally independent, the only way services communicate is through their published interfaces.
    \item Component Scalability - When a single component becomes overloaded, additional instances of the component can be deployed separately.
    \item Distributed development - Separate teams may work on different modules of the system simultaneously if they agree on the interfaces of modules ahead of time.
\end{itemize}

These advantages boost the system's ability to evolve by allowing it to be disassembled into subsystems composed primarily of loosely coupled services.

\paragraph{}

In a microservice architecture, subsystems can fully evolve independently of one another, as long they don't share services or components.
Because services are tied to one another via their interfaces, they can only evolve separately if their contracts remain consistent.
It is often advantageous to have the capacity to change service contracts independently,
particularly in agile workflows and the early phases of development, where contracts are prone to frequent changes.

\section{Problem Statement} % (fold)
\label{sec:problem_statement}

The main challenge in developing applications based on microservices-based architectures is the lack of mechanisms for evaluating the safety of service contract updates.

\paragraph{}

In a monolithic application, interactions between different system components are conducted through function calls.
In a distributed system, however, each individual component has to communicate with other components across the network,
without the same guarantees.

In a regular program, refactoring a function definition can be done quickly with the aid of an IDE.
Distributed systems do not have equivalent tools to detect mismatches between endpoints and their consumers.
Developers are left with the burden of manually tracking down and refactoring the system's dependencies across all consumers and producers;
there is a lack of tools that alleviate this problem.

\paragraph{}

With a rising number of separate services and their interactions, contract administration and service integration become progressively more challenging.
Software engineering guidelines assume a scenario of frequent service implementation changes and few interface or contract modifications;
in the context of iterative approaches to software development, such as agile, contracts are typically modified as frequently as implementation details.

Preserving the integrity of microservice architectures is a daunting task that can only be accomplished by the most diligent teams, for the following reasons:

\paragraph{- The ramifications of changing a contract are discovered after the fact:}

Syntactic or semantic changes in service contracts are usually silent because services evolve independently and contracts are rarely formally specified or even documented.
Without explicit contract verification at deployment time, the only symptoms that the system is broken are runtime errors or unexpected behaviors.
It is paramount to have contract management methods and a common contract specification in place, as this allows the safety of service-based architectures to be checked a priori.

\paragraph{- The safety of a deployment operation is unknown:}

Established platforms such as Kubernetes cannot ensure the safety of deployment operations since they rely primarily on service level meta-information (such as version IDs) to manage deployments.
To assure the safety of deployment tasks, a new model for such environments is required,
one that uses generic meta-information from previously deployed services, adopts expanded type-based contracts and employs a compatibility relation on service contracts.

\paragraph{- A contract change, entails the redeployment and downtime of its consumers:}

Modifications to type definitions in a producer module have immediate
impact on the whole system by requiring downtime on all consumer modules.
Deployments of “compatible” modules should not disrupt the system soundness by imposing the redeployment of services.
Instead, one service should be able to be replaced while the remainder of the system remains operational.

\section{Contributions} % (fold)
\label{sec:contributions}

We present a microservice management system that aims to solve the problems stated above.

\paragraph{}

The challenge of insuring the safety of contract evolutions will be handled by defining and adopting an API description language, as well as implementing a schema
registry that offers a pre-flight safety check procedure that verifies whether two service contracts are compatible before
deployment.
The compatibility mechanism is akin to typechecking procedures in compiled languages and is applied upon a
pair of service contracts, which are typically two distinct versions of the same service. The aforementioned mechanism preserves
a distributed system's correct behavior by rejecting additions or modifications that would threaten it and by informing the programmer
of all the repercussions of the deployment, such as consumer services that are outdated.

\paragraph{}

To ensure that changes in producers contracts do not require the consumers upgrade and redeployment, contract evolution
will be supported at runtime by a generated proxy component that dynamically adapts the data exchanged between services.
The proxy adapter is capable intercepting and evaluating all the outgoing and incoming TCP requests to determine whether
they should be left unchanged or how they should be transformed using an adaptation protocol that extends the compatibility mechanism discussed above.
We allow for addition, removal, renaming and type migration of data fields of a producer module without breaking or upgrading consumer services.
More complex changes in data fields, such as altering the format of a date, will be permitted through the use of user-defined adaption functions.
We also support the modifications on the signatures of WebAPIs, such as changing a path parameter in a HTTP endpoint to a query parameter.
The contract translation specification between two compatible contracts can be audited and modified by a programmer before it is employed by an adapter.

\paragraph{}

Our work will make the following contributions, which will be discussed in depth in Chapter 4:

\begin{itemize}
    \item A compatibility verification on service contracts that determines whether or not messages may be exchanged without data loss;
    \item Definition of a core language for the specification of contract compatibility corrections;
    \item Implementation of a robust type-directed adaptation protocol for service contracts;
    \item Implementation of a service registry tasked with holding service contracts, tracking service dependencies;
    and providing a lightweight preflight safety check procedure for deployment/un-deployment operations;
    \item Implementation of a benchmark platform to compare the solution to existing ones.
\end{itemize}

\section{Document Scructure} % (fold)
\label{sec:document_structure}

The following sections of the document start with a review of the main the concepts,
techniques and applications behind our approach (Section 2).
Sequent sections, first introduce a summary of the key related work (Section 3), next a more detailed descrip-
tion of the proposed approach to the problem stated (Section 4) and conclude with the
expected phases of the work plan (Section 5).